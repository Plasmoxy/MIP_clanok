% Sebastián Petrík MIP 2020/2021
\documentclass[11pt,slovak,a4paper]{article}
\usepackage[slovak]{babel}
\usepackage[utf8]{inputenc}
\usepackage{graphicx}
\usepackage{url}
%\usepackage{hyperref}
\usepackage{cite}

\pagestyle{headings}
\title{Čaosvý manažment\thanks{Semestrálny projekt v predmete Metódy inžinierskej práce, ak. rok 2020/21, vedenie: Zuzana Špitálová}}
\author{Sebastián Petrík\\[2pt]
	{\small Slovenská technická univerzita v Bratislave}\\
	{\small Fakulta informatiky a informačných technológií}\\
	{\small \texttt{xpetriks1@stuba.sk}}
	}
\date{\small 3. november 2020}

\begin{document}
\maketitle

\begin{abstract}
Svojím článkom by som sa chcel zamerať na problematiku časového manažmentu (time management) a plánovania úloh ako súčasť individuálnej a skupinovej organizácie informácií. Plánujem preskúmať základné znaky tejto disciplíny, jej rôzne aspekty, výhody, nevýhody a dôležitosť pre rôzne skupiny ľudí. Budem sa venovať zhodnoteniu aktuálneho stavu výskumu v tejto oblasti.

Ďalej preskúmam možnosti nástrojov spojených s organizáciou času a informácií (digitálne aj klasické), ich kategórie, rozdielne aj podobné vlastnosti. Chcem opísať aktuálny stav digitálnych nástrojov na organizáciu času, spôsoby, akými ich ľudia využívajú a kategórie používateľov týchto nástrojov (zdroj č. 2). Zároveň by som rád preskúmal vplyv časového manažmentu v edukačnom prostredí (vplyv na študenta vysokej školy, jeho produktívnosť a psychiku).

\end{abstract}
\newpage

\section{Úvod}


\section{Nejaká časť} \label{nejaka}



\section{Záver} \label{zaver} % prípadne iný variant názvu


\bibliography{literatura}
\bibliographystyle{alpha}
\end{document}
