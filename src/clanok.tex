% Sebastián Petrík MIP ZS 2020/2021
% FIIT STU BA

\documentclass[10pt,slovak,a4paper]{article}
\usepackage[slovak]{babel}
\usepackage[utf8]{inputenc}
\usepackage{graphicx}
\usepackage{url}
%\usepackage{hyperref}
\usepackage{cite}
\usepackage{indentfirst}
\usepackage{multicol}

\pagestyle{headings}
\title{Čaosvý manažment a organizácia informácií\thanks{Semestrálny projekt v predmete Metódy inžinierskej práce, ak. rok 2020/21, vedenie: Zuzana Špitálová}}
\author{Sebastián Petrík\\[2pt]
	{\small Slovenská technická univerzita v Bratislave}\\
	{\small Fakulta informatiky a informačných technológií}\\
	{\small \texttt{xpetriks1@stuba.sk}}
	}
\date{\small 3. november 2020}

\begin{document}
\maketitle

\begin{abstract}
Svojím článkom by som sa chcel zamerať na problematiku časového manažmentu (time management) a plánovania úloh ako súčasť individuálnej a skupinovej organizácie informácií. Plánujem preskúmať základné znaky tejto disciplíny, jej rôzne aspekty, výhody, nevýhody a dôležitosť pre rôzne skupiny ľudí. Budem sa venovať zhodnoteniu aktuálneho stavu výskumu v tejto oblasti.

Ďalej preskúmam možnosti nástrojov spojených s organizáciou času a informácií (digitálne aj klasické), ich kategórie, rozdielne aj podobné vlastnosti. Chcem opísať aktuálny stav digitálnych nástrojov na organizáciu času, spôsoby, akými ich ľudia využívajú a kategórie používateľov týchto nástrojov (zdroj č. 2). Zároveň by som rád preskúmal vplyv časového manažmentu v edukačnom prostredí (vplyv na študenta vysokej školy, jeho produktívnosť a psychiku).

\end{abstract}
\newpage

\section{Úvod}

	Časový manažment, plánovanie úloh a organizácia informácií sú pojmy, s ktorými sa často stretne človek prakticky v každej oblasti. Týkajú sa nielen pracovníkov vo firme (vykonávajúcich či už psychickú, ale aj fyzickú prácu), ale aj hlavne študentov vysokých a stredných škôl.
	
	Človek je v dnešnom svete zahltený množstvom informácií a úlohami, ktoré je potrebné vykonať v určitom časovom intervale. Prirodzene preto vzniká potreba vytvoriť si určitý systém organizácie.
	
	Cieľom je nielen zabezpečiť dokončenie úloh v termíne, ale aj do určitého stupňa eliminovať zábudlivosť, centralizovať dôležité informácie a termíny, zlepšiť kvalitu práce, zvýšiť efektivitu a zbaviť sa zbytočného stresu.
	
\section{Stav v oblasti}
	
	Napriek tomu, že najmä v pracovných prostrediach pracujúcich s informáciami sú pocity uponáhľanosti a prerušenia súvislej práce vplyvom určitých vyrušení (tzv. fragmentácia práce \cite{NoTask}) pomerne časté a veľmi rozšírené, existuje prekvapivo malé množstvo akademického výskumu venujúceho sa manažmentu a plánovaniu úloh \cite{Franssila}. 
	
	Väčšina štúdií v tomto obore \cite{Franssila,NoTask,Blandford} sa venuje práve fragmentácii práce a analýze pracovných prostredí a praktík časového manažmentu, ktoré tieto prostredia využívajú.
	
	Niektoré štúdie \cite{Franssila, Blandford, Haraty} však skúmajú nástroje, ktoré je možné využiť na organizáciu času a úloh. Taktiež sa venujú kategorizácii používateľov podľa nimi rôznych preferovaných spôsobov organizácie úloh, ale aj dôsledok existencie rôznych kategórií používateľov na dizajn nástrojov určených pre tento problém \cite{Haraty}.
	
	Tento problém je zároveň pomerne dôležitý v edukačnom prostredí. Niektoré psychologické štúdie \cite{Macan} sa venujú vplyvu časového manažmentu na študentov a ich psychiku.
	
\section{Problematika}

	S problematikou časového manažmentu sa dajú spojiť rôzne pojmy:
	\begin{itemize}
		\item časový manažment
		\item plánovanie úloh
		\item organizácia informácií
	\end{itemize}

	Franssila vo svojej štúdii\cite{Franssila} vysvetľuje, že koncepty úlohového manažmentu, plánovania úloh, plánovania aktivít a časového manažmentu sú často v akademickej literatúre využívané ako synonymá, referujúce na rôzne kategórie aktivít spojených s plánovaním a načasovaním úloh.\cite{Franssila}.
	
	Je dôležité si uvedomiť, že tieto rôzne pojmy sa venujú spoločnému problému, a preto sú často využívané spolu. Tento fakt si môžeme všimnúť napríklad pri skúmaní funkcionality nástrojov spojených s touto problematikou, kde v jednom nástroji často nachádzame funkcie pre správu času, ale aj úloh a všeobecných informácií.
	
	\subsection{Časový manažment}
	
		Časový manažment by sme mohli definovať ako súbor aktivít spojených s organizáciou času, ktorý má individuál alebo skupina prístupný v určitom časovom intervale.
		
		Tento časový interval môže byť vnímaný z rôzneho hľadiska - môže byť vymedzený napr. základnými potrebami (spánok, stravovanie), určitými termínmi (napr. potreba dokončiť projekt do termínu - vzniká potreba manažovať si tento vymedzený čas), a pod.
		
	\subsection{Plánovanie úloh}
	
		Plánovanie úloh je možné vnímať ako priamu súčasť časového manažmentu, teda ako konkrétne dedikovanie jednotlivých časových segmentov z určeného časového intervalu na ruešenie určitých úloh.
		
		Franssila uvádza, že hlavné aktivity spojené s plánovaním úloh sú \cite{Franssila}:
		\begin{itemize}
			\item plánovanie
			\item prioritizácia
			\item tvorba zoznamov
		\end{itemize}
	\subsection{Organizácia informácií}
	
	\subsection{Fragmentácia práce}
	% definicia
	\subsection{Vyrušenia}
	% typy, mozu byt benefitial, inkubacia problemu
	% 

\section{Nástroje na organizáciu času a informácií}
	% Fransilla introduction~interface design
	\subsection{Význam}
	\subsection{Klasické}
	\subsection{Digitálne}
	\subsection{Kategórie používateľov}

\section{Dôležitosť časového manažmentu}
\section{Záver}

\bibliography{literatura}
\bibliographystyle{plain}
\end{document}
