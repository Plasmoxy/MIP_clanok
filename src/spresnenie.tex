Svojím článkom by som sa chcel zamerať na problematiku časového manažmentu (time management) a plánovania úloh ako súčasť individuálnej a skupinovej organizácie informácií. Plánujem preskúmať základné znaky tejto disciplíny, jej rôzne aspekty, výhody, nevýhody a dôležitosť pre rôzne skupiny ľudí. Budem sa venovať zhodnoteniu aktuálneho stavu výskumu v tejto oblasti. Ďalej preskúmam možnosti nástrojov spojených s organizáciou času a informácií (digitálne aj klasické), ich kategórie, rozdielne aj podobné vlastnosti. Chcem opísať aktuálny stav digitálnych nástrojov na organizáciu času, spôsoby, akými ich ľudia využívajú a kategórie používateľov týchto nástrojov (zdroj č. 2). Zároveň by som rád preskúmal vplyv časového manažmentu v edukačnom prostredí (vplyv na študenta vysokej školy, jeho produktívnosť a psychiku). 